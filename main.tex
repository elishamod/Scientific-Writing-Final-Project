%% AASTeX v7+ calls the following external packages:
%% times, hyperref, ifthen, hyphens, longtable, xcolor, 
%% bookmarks, array, rotating, ulem, and lineno 
\documentclass[twocolumn]{aastex701}

\usepackage{booktabs}
\usepackage[table]{xcolor}
\usepackage{colortbl}
\usepackage{natbib}
\usepackage{amsmath}

\newcommand{\vdag}{(v)^\dagger}
\newcommand\aastex{AAS\TeX}
\newcommand\latex{La\TeX}

\begin{document}

\title{A Robust, Novel, and Comprehensive Insight: Leveraging a Framework to Utilize Buzzwords in Scientific Abstracts}

\author[orcid=0000-0003-4105-3443]{Elisha Modelevsky}
\affiliation{Racah Institute of Physics, Hebrew University of Jerusalem}
\email[show]{elisha.modelevsky@mail.huji.ac.il}  

%% Mark off the abstract in the ``abstract'' environment. 
\begin{abstract}

This paper...

\end{abstract}

% \keywords{\uat{Galaxies}{573}}
%% You can use the \uat command to link your UAT concepts back its source.
% \keywords{\uat{Galaxies}{573} --- \uat{Cosmology}{343} --- \uat{High Energy astrophysics}{739} --- \uat{Interstellar medium}{847} --- \uat{Stellar astronomy}{1583} --- \uat{Solar physics}{1476}}

\section{Introduction} 

ChatGPT was introduced on 30 November 2022 by OpenAI as a revolutionary large language model (LLM), capable of understanding and generating human-like text \citep{Roumeliotis2023}.
Since its release, many other LLMs with similar capabilities have emerged, and these models are now widely used for various tasks, including writing and coding \citep{Minaee2024}.
The influence of LLMs on science is an ongoing discussion -- while they can be very useful and efficient for many stages in the scientific process, some researchers express concerns about the potential for misuse and the impact on scientific integrity \citep{Zhang2025}.

Another aspect of science where the footprint of LLMs is very apparent is scientific writing.
Previous studies have demonstrated linguistic shifts in scientific language;
notably, some studies have found a significant increase in the use of certain words and phrases since the introduction of ChatGPT \citep{Kobak2025,Bao2025,Juzek2024}.
An interesting phenomenon highlighted by \citet{Lin2025} is the ``equalization'' of scientific language across countries of origin, characterized by a reduction in linguistic differences between native and non-native English-speaking authors.

Prior studies have investigated linguistic shifts in general scientific writing and in select disciplines \citep{Xu2024,Kobak2025,Bao2025}, but to our knowledge, a systematic analysis focusing specifically on astrophysics has not yet been conducted.
This work aims to fill this gap by examining the change in vocubalary in astrophysics abstracts since the introduction of ChatGPT, and comparing it to other branches of physics.
We also investigate whether different subfields within astrophysics have been affected differently.


\section{Methods}

\begin{enumerate}
    \item A list of words that are commonly cited as overused by LLMs.
    \item We collect our data using the arXiv API.
    \item Due to API limitations, we take 250 (or less) articles per month.
    \item We bin the data by yearly quarters.
    \item To see if usage has significantly changed since ChatGPT, we apply a z-test on word frequency before and after.
\end{enumerate}

\section{Results}

\begin{table*}[h]
\centering
\scriptsize
\setlength{\tabcolsep}{3pt}
\begin{tabular}{|l|c|c|c|c|c|c|c|c|c|c|c|}
\toprule
 & astro-ph.CO & astro-ph.EP & astro-ph.GA & astro-ph.HE & astro-ph.IM & astro-ph.SR & astro-ph & cond-mat & hep & nucl & cs \\
Word &  &  &  &  &  &  &  &  &  &  &  \\
\midrule
additionally & \cellcolor{yellow!53} 2.68 & \cellcolor{yellow!38} 1.90 & \cellcolor{yellow!46} 2.31 & \cellcolor{yellow!46} 2.30 & \cellcolor{yellow!38} 1.90 & \cellcolor{yellow!44} 2.20 & \cellcolor{yellow!37} 1.87 & \cellcolor{yellow!46} 2.31 & \cellcolor{yellow!72} 3.63 & \cellcolor{yellow!65} 3.27 & \cellcolor{yellow!47} 2.38 \\
albeit &  &  &  &  &  &  &  & \cellcolor{yellow!35} 1.77 & \cellcolor{yellow!38} 1.91 &  &  \\
boast &  &  &  &  &  &  &  &  &  &  & \cellcolor{yellow!100} 7.00 \\
capabilities &  &  & \cellcolor{yellow!28} 1.44 &  & \cellcolor{yellow!27} 1.36 & \cellcolor{yellow!28} 1.42 & \cellcolor{yellow!25} 1.27 &  &  &  & \cellcolor{yellow!52} 2.64 \\
change &  & \cellcolor{yellow!17} 0.89 &  &  &  &  & \cellcolor{yellow!17} 0.89 &  &  & \cellcolor{yellow!23} 1.15 &  \\
comprehensive & \cellcolor{yellow!39} 1.97 & \cellcolor{yellow!36} 1.81 & \cellcolor{yellow!38} 1.90 & \cellcolor{yellow!41} 2.07 & \cellcolor{yellow!46} 2.30 & \cellcolor{yellow!32} 1.63 & \cellcolor{yellow!39} 1.95 & \cellcolor{yellow!40} 2.00 & \cellcolor{yellow!42} 2.14 & \cellcolor{yellow!41} 2.07 & \cellcolor{yellow!57} 2.90 \\
critical & \cellcolor{yellow!24} 1.23 & \cellcolor{yellow!23} 1.15 & \cellcolor{yellow!25} 1.26 & \cellcolor{yellow!24} 1.20 & \cellcolor{yellow!23} 1.17 & \cellcolor{yellow!23} 1.17 &  &  & \cellcolor{yellow!24} 1.22 &  & \cellcolor{yellow!36} 1.83 \\
crucial & \cellcolor{yellow!34} 1.70 & \cellcolor{yellow!32} 1.64 & \cellcolor{yellow!38} 1.91 & \cellcolor{yellow!33} 1.68 & \cellcolor{yellow!36} 1.83 & \cellcolor{yellow!34} 1.70 & \cellcolor{yellow!34} 1.74 & \cellcolor{yellow!27} 1.35 & \cellcolor{yellow!31} 1.55 & \cellcolor{yellow!40} 2.01 & \cellcolor{yellow!48} 2.43 \\
delv & \cellcolor{yellow!100} 5.67 & \cellcolor{yellow!100} 7.50 & \cellcolor{yellow!80} 4.00 & \cellcolor{yellow!100} 8.00 & \cellcolor{yellow!100} 5.80 & \cellcolor{yellow!100} 9.00 & \cellcolor{yellow!80} 4.00 & \cellcolor{yellow!100} 28.00 & \cellcolor{yellow!100} 6.33 &  & \cellcolor{yellow!100} 6.90 \\
discover &  & \cellcolor{yellow!17} 0.87 & \cellcolor{yellow!23} 1.19 & \cellcolor{yellow!22} 1.10 &  &  &  & \cellcolor{yellow!24} 1.24 &  &  & \cellcolor{yellow!24} 1.21 \\
driven by & \cellcolor{yellow!25} 1.25 & \cellcolor{yellow!25} 1.26 &  & \cellcolor{yellow!25} 1.29 &  &  &  & \cellcolor{yellow!25} 1.28 &  &  & \cellcolor{yellow!53} 2.69 \\
effectively & \cellcolor{yellow!31} 1.55 &  & \cellcolor{yellow!41} 2.07 & \cellcolor{yellow!30} 1.54 & \cellcolor{yellow!36} 1.83 & \cellcolor{yellow!30} 1.52 & \cellcolor{yellow!31} 1.57 & \cellcolor{yellow!27} 1.38 & \cellcolor{yellow!43} 2.16 & \cellcolor{yellow!35} 1.78 & \cellcolor{yellow!41} 2.09 \\
endeavor &  &  &  &  &  &  &  &  &  & \cellcolor{yellow!68} 3.40 &  \\
enhance & \cellcolor{yellow!28} 1.41 & \cellcolor{yellow!24} 1.20 &  &  & \cellcolor{yellow!41} 2.07 &  & \cellcolor{yellow!29} 1.48 & \cellcolor{yellow!22} 1.14 & \cellcolor{yellow!25} 1.29 & \cellcolor{yellow!27} 1.40 & \cellcolor{yellow!78} 3.93 \\
exceptional & \cellcolor{yellow!33} 1.67 &  & \cellcolor{yellow!31} 1.59 & \cellcolor{yellow!31} 1.56 & \cellcolor{yellow!40} 2.02 & \cellcolor{yellow!29} 1.48 & \cellcolor{yellow!30} 1.54 & \cellcolor{yellow!26} 1.32 &  & \cellcolor{yellow!57} 2.88 & \cellcolor{yellow!57} 2.86 \\
find & \cellcolor{yellow!21} 1.08 & \cellcolor{yellow!21} 1.07 & \cellcolor{yellow!21} 1.06 & \cellcolor{yellow!21} 1.07 & \cellcolor{yellow!22} 1.11 & \cellcolor{yellow!21} 1.08 & \cellcolor{yellow!21} 1.05 & \cellcolor{yellow!23} 1.16 &  & \cellcolor{yellow!23} 1.17 & \cellcolor{yellow!24} 1.21 \\
foster &  &  &  &  &  &  &  &  & \cellcolor{yellow!100} 7.67 & \cellcolor{yellow!66} 3.33 & \cellcolor{yellow!75} 3.75 \\
framework & \cellcolor{yellow!29} 1.47 & \cellcolor{yellow!26} 1.33 & \cellcolor{yellow!27} 1.35 & \cellcolor{yellow!27} 1.35 & \cellcolor{yellow!33} 1.69 & \cellcolor{yellow!26} 1.30 & \cellcolor{yellow!29} 1.49 & \cellcolor{yellow!29} 1.48 & \cellcolor{yellow!26} 1.30 & \cellcolor{yellow!25} 1.29 & \cellcolor{yellow!30} 1.54 \\
groundbreaking &  &  &  &  &  &  & \cellcolor{yellow!60} 3.00 & \cellcolor{yellow!100} 6.50 &  &  & \cellcolor{yellow!100} 5.17 \\
help &  & \cellcolor{yellow!24} 1.20 &  &  &  &  &  &  &  &  & \cellcolor{yellow!17} 0.86 \\
in-depth &  &  &  &  &  & \cellcolor{yellow!35} 1.75 &  & \cellcolor{yellow!44} 2.21 & \cellcolor{yellow!60} 3.00 &  &  \\
innovative &  & \cellcolor{yellow!45} 2.26 &  &  & \cellcolor{yellow!26} 1.33 &  &  & \cellcolor{yellow!80} 4.00 &  & \cellcolor{yellow!38} 1.92 & \cellcolor{yellow!87} 4.37 \\
insight & \cellcolor{yellow!50} 2.53 & \cellcolor{yellow!36} 1.84 & \cellcolor{yellow!46} 2.30 & \cellcolor{yellow!47} 2.35 & \cellcolor{yellow!35} 1.77 & \cellcolor{yellow!50} 2.53 & \cellcolor{yellow!38} 1.92 & \cellcolor{yellow!39} 1.98 & \cellcolor{yellow!46} 2.30 & \cellcolor{yellow!37} 1.89 & \cellcolor{yellow!45} 2.29 \\
intricate & \cellcolor{yellow!77} 3.86 & \cellcolor{yellow!63} 3.17 & \cellcolor{yellow!100} 8.17 & \cellcolor{yellow!100} 9.60 & \cellcolor{yellow!100} 13.33 & \cellcolor{yellow!61} 3.07 & \cellcolor{yellow!100} 5.71 & \cellcolor{yellow!49} 2.49 & \cellcolor{yellow!78} 3.90 & \cellcolor{yellow!100} 9.80 & \cellcolor{yellow!100} 6.28 \\
leverag & \cellcolor{yellow!89} 4.47 & \cellcolor{yellow!60} 3.03 & \cellcolor{yellow!91} 4.58 & \cellcolor{yellow!100} 5.21 & \cellcolor{yellow!73} 3.66 & \cellcolor{yellow!85} 4.26 & \cellcolor{yellow!67} 3.38 & \cellcolor{yellow!70} 3.50 & \cellcolor{yellow!90} 4.52 & \cellcolor{yellow!100} 5.67 & \cellcolor{yellow!53} 2.67 \\
look &  &  &  &  &  & \cellcolor{yellow!26} 1.30 &  &  &  &  &  \\
make &  &  &  & \cellcolor{yellow!15} 0.77 &  &  &  & \cellcolor{yellow!16} 0.80 &  & \cellcolor{yellow!16} 0.84 & \cellcolor{yellow!16} 0.83 \\
need &  &  &  &  &  &  & \cellcolor{yellow!22} 1.11 &  &  &  & \cellcolor{yellow!23} 1.17 \\
notabl & \cellcolor{yellow!52} 2.60 & \cellcolor{yellow!40} 2.01 & \cellcolor{yellow!57} 2.89 & \cellcolor{yellow!56} 2.82 & \cellcolor{yellow!49} 2.49 & \cellcolor{yellow!43} 2.18 & \cellcolor{yellow!39} 1.99 & \cellcolor{yellow!58} 2.92 & \cellcolor{yellow!47} 2.38 & \cellcolor{yellow!92} 4.63 & \cellcolor{yellow!54} 2.70 \\
novel & \cellcolor{yellow!26} 1.34 & \cellcolor{yellow!27} 1.39 & \cellcolor{yellow!31} 1.55 & \cellcolor{yellow!30} 1.50 & \cellcolor{yellow!27} 1.39 & \cellcolor{yellow!29} 1.48 & \cellcolor{yellow!26} 1.33 &  & \cellcolor{yellow!24} 1.22 & \cellcolor{yellow!28} 1.41 & \cellcolor{yellow!27} 1.38 \\
paradigm & \cellcolor{yellow!26} 1.30 & \cellcolor{yellow!32} 1.64 &  &  &  &  & \cellcolor{yellow!32} 1.60 &  &  & \cellcolor{yellow!32} 1.64 & \cellcolor{yellow!30} 1.53 \\
paramount &  &  & \cellcolor{yellow!53} 2.67 &  &  &  &  & \cellcolor{yellow!55} 2.75 &  &  & \cellcolor{yellow!34} 1.73 \\
particularly & \cellcolor{yellow!33} 1.69 & \cellcolor{yellow!30} 1.53 & \cellcolor{yellow!34} 1.73 & \cellcolor{yellow!34} 1.72 & \cellcolor{yellow!32} 1.65 & \cellcolor{yellow!27} 1.40 & \cellcolor{yellow!31} 1.55 & \cellcolor{yellow!32} 1.63 & \cellcolor{yellow!28} 1.45 & \cellcolor{yellow!36} 1.84 & \cellcolor{yellow!50} 2.51 \\
pivotal & \cellcolor{yellow!61} 3.08 & \cellcolor{yellow!100} 5.22 & \cellcolor{yellow!75} 3.77 & \cellcolor{yellow!40} 2.00 & \cellcolor{yellow!94} 4.73 & \cellcolor{yellow!50} 2.50 & \cellcolor{yellow!67} 3.36 & \cellcolor{yellow!64} 3.22 & \cellcolor{yellow!74} 3.73 & \cellcolor{yellow!78} 3.93 & \cellcolor{yellow!100} 6.35 \\
primarily & \cellcolor{yellow!27} 1.35 & \cellcolor{yellow!30} 1.50 & \cellcolor{yellow!33} 1.67 & \cellcolor{yellow!31} 1.59 &  & \cellcolor{yellow!31} 1.57 & \cellcolor{yellow!27} 1.37 & \cellcolor{yellow!45} 2.27 & \cellcolor{yellow!39} 1.98 & \cellcolor{yellow!46} 2.33 & \cellcolor{yellow!51} 2.57 \\
promising & \cellcolor{yellow!27} 1.38 & \cellcolor{yellow!26} 1.34 &  &  & \cellcolor{yellow!23} 1.18 &  & \cellcolor{yellow!27} 1.39 & \cellcolor{yellow!27} 1.35 & \cellcolor{yellow!25} 1.27 & \cellcolor{yellow!30} 1.54 & \cellcolor{yellow!35} 1.79 \\
remarkable & \cellcolor{yellow!29} 1.49 &  & \cellcolor{yellow!29} 1.48 &  &  &  &  &  &  &  & \cellcolor{yellow!57} 2.88 \\
rigorous & \cellcolor{yellow!29} 1.48 &  &  &  &  &  &  &  & \cellcolor{yellow!29} 1.47 &  &  \\
robust & \cellcolor{yellow!27} 1.38 & \cellcolor{yellow!24} 1.20 & \cellcolor{yellow!27} 1.36 & \cellcolor{yellow!25} 1.29 & \cellcolor{yellow!29} 1.46 & \cellcolor{yellow!23} 1.19 & \cellcolor{yellow!24} 1.22 & \cellcolor{yellow!26} 1.30 & \cellcolor{yellow!28} 1.41 & \cellcolor{yellow!33} 1.68 & \cellcolor{yellow!37} 1.86 \\
scalable & \cellcolor{yellow!44} 2.22 &  &  &  & \cellcolor{yellow!36} 1.83 &  & \cellcolor{yellow!47} 2.39 & \cellcolor{yellow!40} 2.03 & \cellcolor{yellow!84} 4.20 &  & \cellcolor{yellow!26} 1.30 \\
show & \cellcolor{yellow!18} 0.94 &  &  & \cellcolor{yellow!19} 0.95 &  &  & \cellcolor{yellow!18} 0.94 & \cellcolor{yellow!18} 0.93 & \cellcolor{yellow!18} 0.91 &  & \cellcolor{yellow!17} 0.85 \\
significant & \cellcolor{yellow!24} 1.21 & \cellcolor{yellow!24} 1.22 & \cellcolor{yellow!24} 1.21 & \cellcolor{yellow!24} 1.23 & \cellcolor{yellow!27} 1.38 & \cellcolor{yellow!23} 1.15 & \cellcolor{yellow!24} 1.22 & \cellcolor{yellow!32} 1.61 & \cellcolor{yellow!28} 1.43 & \cellcolor{yellow!27} 1.36 & \cellcolor{yellow!37} 1.86 \\
small &  & \cellcolor{yellow!18} 0.93 & \cellcolor{yellow!18} 0.91 & \cellcolor{yellow!17} 0.87 & \cellcolor{yellow!17} 0.89 &  & \cellcolor{yellow!17} 0.85 & \cellcolor{yellow!17} 0.87 &  &  &  \\
state-of-the-art &  & \cellcolor{yellow!27} 1.36 &  &  & \cellcolor{yellow!24} 1.23 & \cellcolor{yellow!27} 1.39 &  &  & \cellcolor{yellow!27} 1.38 & \cellcolor{yellow!24} 1.23 & \cellcolor{yellow!25} 1.25 \\
subsequently & \cellcolor{yellow!32} 1.62 &  &  &  & \cellcolor{yellow!32} 1.60 & \cellcolor{yellow!28} 1.45 & \cellcolor{yellow!28} 1.42 &  &  &  & \cellcolor{yellow!37} 1.87 \\
systematic & \cellcolor{yellow!23} 1.15 &  &  &  &  &  & \cellcolor{yellow!23} 1.16 & \cellcolor{yellow!25} 1.26 & \cellcolor{yellow!23} 1.19 & \cellcolor{yellow!24} 1.20 & \cellcolor{yellow!31} 1.55 \\
thereby &  &  & \cellcolor{yellow!37} 1.87 &  & \cellcolor{yellow!32} 1.60 &  &  & \cellcolor{yellow!30} 1.53 & \cellcolor{yellow!31} 1.57 & \cellcolor{yellow!33} 1.67 & \cellcolor{yellow!40} 2.04 \\
thoroughly &  &  &  &  & \cellcolor{yellow!44} 2.21 &  &  &  &  &  &  \\
uncover & \cellcolor{yellow!54} 2.74 & \cellcolor{yellow!30} 1.53 & \cellcolor{yellow!48} 2.41 & \cellcolor{yellow!42} 2.10 &  & \cellcolor{yellow!32} 1.63 &  & \cellcolor{yellow!31} 1.56 & \cellcolor{yellow!35} 1.79 &  & \cellcolor{yellow!37} 1.88 \\
unveil & \cellcolor{yellow!45} 2.28 & \cellcolor{yellow!37} 1.88 & \cellcolor{yellow!34} 1.71 &  &  &  & \cellcolor{yellow!35} 1.76 & \cellcolor{yellow!32} 1.63 &  &  &  \\
utiliz & \cellcolor{yellow!49} 2.45 & \cellcolor{yellow!32} 1.61 & \cellcolor{yellow!54} 2.72 & \cellcolor{yellow!48} 2.44 & \cellcolor{yellow!35} 1.79 & \cellcolor{yellow!49} 2.47 & \cellcolor{yellow!34} 1.74 & \cellcolor{yellow!36} 1.81 & \cellcolor{yellow!49} 2.49 & \cellcolor{yellow!49} 2.46 & \cellcolor{yellow!37} 1.85 \\
valuable & \cellcolor{yellow!43} 2.17 & \cellcolor{yellow!35} 1.75 & \cellcolor{yellow!44} 2.23 & \cellcolor{yellow!43} 2.17 & \cellcolor{yellow!43} 2.16 & \cellcolor{yellow!43} 2.19 & \cellcolor{yellow!35} 1.79 & \cellcolor{yellow!65} 3.28 & \cellcolor{yellow!50} 2.52 & \cellcolor{yellow!38} 1.94 & \cellcolor{yellow!47} 2.39 \\
vital &  & \cellcolor{yellow!32} 1.65 & \cellcolor{yellow!35} 1.76 & \cellcolor{yellow!32} 1.63 & \cellcolor{yellow!33} 1.68 & \cellcolor{yellow!31} 1.57 &  & \cellcolor{yellow!36} 1.84 & \cellcolor{yellow!53} 2.69 & \cellcolor{yellow!46} 2.32 & \cellcolor{yellow!32} 1.61 \\
\bottomrule
\end{tabular}

\caption{Factor increase in word usage after the introduction of ChatGPT, by arXiv category.
Blank cells indicate non-significant changes ($p \geq 0.05$).
The color scale indicates the magnitude of the factor increase, with darker yellow representing larger increases.}
\end{table*}

\section{Conclusions}

\bibliography{references}{}
\bibliographystyle{aasjournalv7}

\end{document}
